\subsection{derive}

Autocurreation function of a quantity \(A(t)\) is in numerical analysis:

\begin{equation}
  \acf{A}{t}{t + \tau} = \lim_{N \rightarrow \infty}\frac{1}{N} \sum_{i = 0}^{N} h(t_i)h(t_i + \tau)
  \label{acf}
\end{equation}

where \(\tau \) is delay/shift time and \(N\) is the number of sampling.
You noticed that variable \(t\) does not appear in the right side because autocorrelation function is a sum of all samples.

If the quantity \(A(t)\) is heat current, \(h(t)\), heat conductivity \(\lambda \) is written. Note that linear response theory works in the region of equilibruim and is required stationarity.

When this situation, we can derive:

\begin{equation}
  \acf{A}{t}{t + \tau}  = \acf{A}{0}{\tau}
\end{equation}

then, we can introduce heat conductivity:

\begin{equation}
  \lambda = \frac{1}{3Vk_B T^2} \int_0^\infty d\tau \acf{h}{0}{\tau}
  \label{heat_conductivity}
\end{equation}

In numerical analysis, Eq.\ref{heat_conductivity} is:

\begin{equation}
  \int_0^\infty d\tau \acf{h}{0}{\tau} = \lim_{N \rightarrow \infty} \sum_{i=0}^N \acf{h}{t_0}{t_i}
  \label{numerical_heat_conductivity}
\end{equation}

\(t_0\) is \(t = 0\). The integration range is limited. Thus, we remove limit to infinity.